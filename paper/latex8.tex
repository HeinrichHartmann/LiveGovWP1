
%
%  $Description: Author guidelines and sample document in LaTeX 2.09$
%
%  $Author: ienne $
%  $Date: 1995/09/15 15:20:59 $
%  $Revision: 1.4 $
%

\documentclass[times, 10pt,twocolumn]{article}
\usepackage{latex8}
\usepackage{times}
\usepackage{hyperref}


%\documentstyle[times,art10,twocolumn,latex8]{article}

%-------------------------------------------------------------------------
% take the % away on next line to produce the final camera-ready version
% \pagestyle{empty}

%-------------------------------------------------------------------------
\begin{document}

\title{MORA - A Sensor Data Analysis Toolkit for Mobile Phones}

\author{Heinrich Hartmann\\
University Koblenz\\ % WeST \\ % IN-F
% Ecublens, 1015 Lausanne, Switzerland\\ Paolo.Ienne@di.epfl.ch\\
% For a paper whose authors are all at the same institution,
% omit the following lines up until the closing ``}''.
% Additional authors and addresses can be added with ``\and'',
% just like the second author.
\and
Christoph Schaefer \\
University Koblenz \\
% First line of institution2 address\\ Second line of institution2 address\\
% SecondAuthor@institution2.com\\
\and
Matthias Thimm \\
University Koblenz \\
}

\maketitle
\thispagestyle{empty}

\begin{abstract}
\end{abstract}

%-------------------------------------------------------------------------
\Section{Introduction} %% MT
- Much sensor data from smartphones available.

- Problem: Extract sensible information from raw data. E.g. for
  - Quantified Self
  - Internet of Things

- Need tools for gathering test data and evaluation of algorithms: MORA.

%-------------------------------------------------------------------------
\Section{MORA Toolkit}

* Goals: What should MORA do?

\SubSection{Overview}
* How are the goals achieved?

* Three separate components

* What is done on the mobile device? What is done on the server?

* Figure: With three components: Mobile - Server - Web

\SubSection{Mobile Sensor Collector}
* Figure: new GUI from Lukas

* Goals and 'Requirements'
  - Correctness of data / good performance
  - View sensor data as events, that are logged / streamed / processed
  - Extensible by analysis plugins
  - Transfer of sensor data to server

* Service + Example GUI (Communication with Intent bindings)
  - Android lifecycle forces collector to be a service
  - Standalone use of service possible and intended

* Available Sensors. Types of sensors. Class Hierarchy.

* Data flow from Sensor Event to Log file,
  in analogy to logging frameworks.
  Use of Threads

* Transfer via HTTP and Streaming

* Configuration options with configuration file and GUI

* Log messages are written to a file

* Extendability with Plugins: Consumer/Producer. Examples:
  - HAR
  - GPS Cache
  - Intent producer

\SubSection{Sensor Storage Server}

* HTTP server (tomcat/Node), that accepts POST request

* Stores in FS and in DB

* Sends 200 Status code if insert in db complete

* JSF Data Exchange Format

* Auto Creation of DB Tables for new sample types (JSON data type)

\SubSection{Web Inspection Tool}
\url{liveandgov.uni-koblenz.de/storage/inspection}

+ Figure: Screenshot mit Bar-Code Anzeige von Activities


An important step that should be considered before any data mining task is a manual data exploration. For this purpose, MORA provides a powerful data inspection tool. Using this web-based tool, both raw data and mining results can be viewed in the browser. Thus, a first overview of the data integrity can be obtained quickly. Especially in combination with mobile devices, occasionally appearing transmission errors can be discovered easily by the user. In addition, plots for all sensor type give a rough measure for the quality of the recorded data. With the help of this presentation we could, for example, discover GPS inaccuracies or frequency deviations. Besides the plots of time series data and map based views of GPS data.

  
  - Simple export of sample bulks as CSV for processing with other applications.
  - Privacy Aspects. Give users control over the data.

* Features:
  - Browser based
  - Zoom
  - Views:
    - Plots of time series data (e.g. accelerometer)
    - Map based views of GPS data
    - Lists of textual data (tags)
    - Bar-Code-like view of Tags
    - Map markers of tags
  - Delete recordings
  - Automatic display of available tables
    -> as text (TEXT) or plot (float)

* Mining output has to be stored in db
  - Can be created on mobile device and inserted into transfer file
  - Or can be done on the server

%-------------------------------------------------------------------------
\Section{Practical Examples}

\SubSection{Human Activity Recognition}
* Decision tree learning with WEKA
* Import classifier as JAVA class into MORA Library

\SubSection{Service Line Detection}
* Web service for SLD
* GPS Samples are gathered via MORA lib
* Query results are inserted back into MORA
* Analysis of classification results via inspection tool

%-------------------------------------------------------------------------
\Section{Related Work}

* FUNF
* SDCF


\nocite{ex1,ex2}
\bibliographystyle{latex8}
\bibliography{latex8}

\end{document}
